\documentclass{homework}

\homework{1}
\date{Tuesday 1/29}
\author{}
\coauthor{}

\begin{document}
\begin{problem}[P.A.1]
  Show that:
  \begin{enumerate}
  \item \((1+i)^4 = -4\)

    \begin{solution}
    \end{solution}

  \item \((1-i)^{-1} - (1+i)^{-1} = i\)

    \begin{solution}
    \end{solution}

  \item \((i-1)^{-4} = -1/4\)

    \begin{solution}
    \end{solution}

  \item \(10 (1+3i)^{-1} = 1-3i\)

    \begin{solution}
    \end{solution}

  \item \((\sqrt 3 + i)^3 = 8i\)

    \begin{solution}
    \end{solution}

  \end{enumerate}
\end{problem}

\begin{problem}[P.A.2]
  Evaluate the following expressions.  Write your answer in
  the form \(a+bi\), in which \(a\) and \(b\) are real.
  \begin{enumerate}
  \item \((1+i)(2+3i)\)

    \begin{solution}
    \end{solution}

  \item \(\frac{2+3i}{1+i}\)

    \begin{solution}
    \end{solution}

  \item \(\Paren{\frac{(2+i)^2}{4-3i}}^2\)

    \begin{solution}
    \end{solution}

  \item \(e^{i\alpha} \overline{e^{\beta}}\), in which \(\alpha\) and
    \(\beta\) are real

    \begin{solution}
    \end{solution}

  \item \(\overline{(1-i) \overline{(2+2i)}}\)

    \begin{solution}
    \end{solution}

  \item \(\abs{2-i}^3\)

    \begin{solution}
    \end{solution}

  \end{enumerate}
\end{problem}

\begin{problem}[P.3.3]
  Let \(X = \begin{bmatrix}X_1&X_2\end{bmatrix} \in \M_{m\times n}\),
  in which \(X_1 \in \M_{m\times n_1}\), \(X_2 \in \M_{m\times n_2}\),
  and \(n_1 + n_2 = n\).  Compute \(X^\T X\) and \(X X^\T\).

  \begin{solution}
  \end{solution}

\end{problem}

\begin{problem}[P.3.7]
  Partition \(M \in \M_n\) as a \(2\times 2\) block matrix
  as in (3.4.5).  If \(D\) is invertible, then the \emph{Schur
    complement of \(D\) in \(M\)} is \(M/D = A - BD^{-1} C\).
  \begin{enumerate}
  \item Show that \(\det M = (\det D)(\det M/D)\).

    \begin{solution}
    \end{solution}

  \item If \(A\) is invertible, show that the determinant of the
    bordered matrix (3.4.11) can be evaluated as
    \begin{align}
      \det
      \begin{bmatrix}
        c      & \vec x^\T \\
        \vec y & A
      \end{bmatrix}
               &= (c - \vec x^\T A^{-1} \vec y) \det A \nonumber \\
               &= c \det A - \vec x^\T (\adj A) \vec y. \label{eq:det-bordered-mat}
    \end{align}
    This is the \emph{Cauchy expansion of the determinant} of a
    bordered matrix.  The formulation \eqref{eq:det-bordered-mat} is
    valid even if \(A\) is not invertible.

    \begin{solution}
    \end{solution}

  \end{enumerate}
\end{problem}

\begin{problem}[P.3.9]
  Let \(M = \begin{bmatrix}A&B\\0&D\end{bmatrix} \in \M_n\)
  be block upper triangular and let \(p\) be a polynomial.  Prove that
  \(p(M) = \begin{bmatrix}p(A)&\star\\0&p(D)\end{bmatrix}\).

  \begin{solution}
  \end{solution}

\end{problem}

\begin{problem}[P.3.16]
  Suppose that \(1 \le r \le \min\set{m,n}\).  Let
  \(X \in \M_{m\times r}\) and \(Y \in \M_{r\times n}\), and suppose
  that \(\rank X = \rank Y = r\).  Explain why there are
  \(X_2 \in \M_{m\times(m-r)}\) and \(Y_2 \in \M_{(n-r)\times n}\) such
  that
  \[
    B = \begin{bmatrix}X&X_2\end{bmatrix} \in \M_m
    \quad\text{and}\quad C = \begin{bmatrix}Y\\Y_2\end{bmatrix} \in
    \M_n
  \]
  are invertible.  Verify that
  \[
    XY = B \begin{bmatrix}I_r&0\\0&0\end{bmatrix} C.
  \]
  What sizes are the zero submatrices?

  \begin{solution}
  \end{solution}

\end{problem}

\begin{problem}[P.3.28]
  If an invertible matrix \(M\) is partitioned as a
  \(2\times2\) block matrix as in (3.4.5), there is a conformally
  partitioned presentation of its inverse:
  \begin{equation}
    M^{-1} =
    \begin{bmatrix}
      (A-BD^{-1}C)^{-1}          & -A^{-1} B(D-CA^{-1}B)^{-1} \\
      -D^{-1}C (A-BD^{-1}C)^{-1} & (D-CA^{-1}B)^{-1}
    \end{bmatrix}, \label{eq:block-inverse}
  \end{equation}
  provided that all the indicated inverses exist.
  \begin{enumerate}
  \item Verify that \eqref{eq:block-inverse} can be
    written as
    \begin{equation}
      M^{-1}
      \begin{bmatrix}
        A^{-1} & 0 \\
        0      & D^{-1}
      \end{bmatrix}
      \begin{bmatrix}
        A  & -B \\
        -C & D
      \end{bmatrix}
      \begin{bmatrix}
        (M/D)^{-1} & 0 \\
        0          & (M/A)^{-1}
      \end{bmatrix}. \label{eq:factor-block-inverse}
    \end{equation}

    \begin{solution}
    \end{solution}

  \item Derive the identity from \eqref{eq:factor-block-inverse}.

    \begin{solution}
    \end{solution}

  \item If all the blocks in \eqref{eq:factor-block-inverse} are
    \(1\times1\) matrices, show that it reduces to
    \[
      M^{-1} = \begin{bmatrix}a&b\\c&d\end{bmatrix}^{-1} =
      \frac{1}{\det M} \begin{bmatrix}d&-b\\-c&a\end{bmatrix}.
    \]

    \begin{solution}
    \end{solution}

  \end{enumerate}
\end{problem}

\begin{problem}[P.3.29]
  Suppose that \(A \in \M_{n\times m}\) and
  \(B \in \M_{m\times n}\).  Use the block matrix
  \[
    \begin{bmatrix}
      I_n & -A \\
      B   & I_m
    \end{bmatrix}
  \]
  to derive the \emph{Sylvester determinant identity}
  \begin{equation}
    \det(I_n + AB) = \det(I_m + BA), \label{eq:sylvester-det-identity}
  \end{equation}
  which relates the determinant of an \(n\times n\) matrix to the
  determinant of an \(m\times m\) matrix.

  \begin{solution}
  \end{solution}

\end{problem}
\end{document}