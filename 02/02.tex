\documentclass{homework}

\homework{2}
\date{Tuesday 2/5}
\author{}
\coauthor{}

\begin{document}
\begin{description}
\item[P.3.17]
  Let \(A \in \M_{m\times n}\).  Use the preceding problems to show
  that \(\rank A = r\) if and only if there are invertible
  \(B \in \M_m\) and \(C \in M_n\) such that
  \[
    A = B
    \begin{bmatrix}
      I_r & 0 \\
      0   & 0
    \end{bmatrix}
    C.
  \]

  \begin{solution}

  \end{solution}

\item[P.4.3]
  Provide details for the following alternative proof of the
  Cauchy--Schwarz inequality in a real inner product space \(V\).  For
  nonzero vectors \(\vec u, \vec v \in V\), consider the function
  \(p(t) = \norm{t \vec u + \vec v}^2\) of a real variable \(t\).
  \begin{enumerate}
  \item Why is \(p(t) \ge 0\) for all real \(t\)?

    \begin{solution}

    \end{solution}

  \item If \(p(t) = 0\) has a real root, why are \(\vec u\) and
    \(\vec v\) linearly dependent?

    \begin{solution}

    \end{solution}

  \item Show that \(p\) is a polynomial
    \[
      p(t) =
      \norm{\vec u}^2 t^2        +
      2 \inner{\vec u, \vec v} t +
      \norm{v}^2
    \]
    of degree \(2\) with real coefficients.

    \begin{solution}

    \end{solution}

  \item If \(\vec u\) and \(\vec v\) are linearly independent, why
    does \(p(t) = 0\) have no real roots?

    \begin{solution}

    \end{solution}

  \item Use the quadratic formula to deduce the Cauchy--Schwarz
    inequality for a real inner product space.

    \begin{solution}

    \end{solution}
  \end{enumerate}

\item[P.4.4]
  Modify the argument in the preceding problem to prove the
  Cauchy--Schwarz inequality in a complex inner product space \(V\).
  Redefine \(p(t) = \norm{t \vec u + e^{i\theta} \vec v}^2\), in which
  \(\theta\) is a real parameter such that
  \(e^{-i\theta} \inner{\vec u, \vec v} = \abs{\inner{u, v}}\).
  \begin{enumerate}
  \item Explain why such a choice of \(\theta\) is possible and why
    \[
      p(t) =
      \norm{\vec u}^2 t^2    +
      2 \abs{\inner{\vec u, \vec v}} t +
      \norm{\vec v}^2,
    \]
    a polynomial of degree \(2\) with real coefficients.

    \begin{solution}

    \end{solution}

  \item If \(\vec u\) and \(\vec v\) are linearly independent, why
    does \(p(t) = 0\) have no real roots?

    \begin{solution}

    \end{solution}

  \item Use the quadratic formula to deduce the Cauchy--Schwarz
    inequality for a complex inner product space.

    \begin{solution}

    \end{solution}
  \end{enumerate}

\item[P.4.10]
  Let \(\VV\) be an \(\FF\)--inner product space and let
  \(\vec u, \vec v \in \VV\).  Prove that \(\vec u\) and \(\vec v\)
  are orthogonal if and only if
  \(\norm{\vec v} \le \norm{c \vec u + \vec v}\) for all
  \(c \in \FF\).  Be sure your proof covers both cases \(\FF = \RR\)
  and \(\FF = \CC\).  Draw a diagram illustrating what this means if
  \(\VV = \RR^2\).

  \begin{solution}

  \end{solution}

\item[P.4.11]
  \begin{enumerate}
  \item Let \(f, g \in C_\RR [0, 1]\) (see Example 4.4.8) be the
    functions depicted in Figure 4.9.  Does there exist a
    \(c \in \RR\) such that \(\norm{f + cg} < \norm f\)?

    \begin{solution}

    \end{solution}

  \item Consider the functions \(f(x) = x(1-x)\) and
    \(g(x) = \sin(2\pi x)\) in \(C_\RR [0, 1]\).  Does there exist a
    \(c \in \RR\) such that \(\norm{f + cg} < \norm f\)?

    \begin{solution}

    \end{solution}
  \end{enumerate}
  \begin{center}
    \pgfplotsset{
      depicted function/.style={
        width=3in,
        height=2in,
        axis lines=left,
        yticklabel style={
          /pgf/number format/.cd,
          fixed,
          precision=3,
        },
        scaled ticks=false,
        font=\small,
      },
    }
    \hfil
    \begin{tikzpicture}
      \begin{axis}[depicted function]
        \addplot[domain=0:1, samples=256] {
          exp(-((x-.5) / .25)^2) *
          (20 - 2.5 * cos(360*24*x))
        };
      \end{axis}
    \end{tikzpicture}
    \hfil
    \begin{tikzpicture}
      \begin{axis}[
        depicted function,
        ytick distance=.005,
        ]
        \addplot[domain=0:1, samples=385] {
          x * (1-x) *
          (1 - cos(360*24*x)) *
          .028 / .5
        };

      \end{axis}
    \end{tikzpicture}
    \hfil
  \end{center}

\item[P.4.23]
  Let \(\VV = P_n\), \(p(z) = \sum_{k=0}^n p_k z^k\) and
  \(q(z) = \sum_{k=0}^n q_k z^k\); define
  \[
    \inner{p, q} = \sum_{i,j=0}^n \frac{p_i \conj{q_j}}{i+j+1}.
  \]
  \begin{enumerate}
  \item Show that
    \(\inner{\cdot, \cdot} \colon \VV \times \VV \to \CC\) is an inner
    product.

    \begin{solution}

    \end{solution}

  \item Deduce that the matrix
    \(A = \begin{bmatrix} (i+j-1)^{-1} \end{bmatrix}\) has the
    property that \(\inner{A \vec x, \vec x} \ge 0\) for all
    \(\vec x \in \CC^n\).

    \begin{solution}

    \end{solution}
  \end{enumerate}


\item[P.5.6*]
  Let \(\vec u_1, \vec u_2, \dots, \vec u_n\) be an orthonormal system
  in an \(\FF\)--inner product space \(\VV\).
  \begin{enumerate}
  \item Prove that
    \[
      \norm{\vec v - \sum_{i=1}^n a_i \vec u_i}^2 =
      \norm{\vec v}^2 -
      \sum_{i=1}^n \abs{\inner{\vec v, \vec u_i}}^2 +
      \sum_{i=1}^n \abs{\inner{\vec v, \vec u_i} - a_i}^2
      \tag{5.9.1}
    \]
    for all \(a_1, a_2, \dots, a_n \in \FF\) and all \(\vec v \in V\).

    \begin{solution}

    \end{solution}

  \item Let \(\vec v \in \VV\) be given and let
    \(\WW = \spn\set{\vec u_1, \vec u_2, \dots, \vec u_n}\).  Prove
    that there is a unique \(\vec x \in \WW\) such that
    \(\norm{\vec v - \vec x} \le \norm{\vec v - \vec w}\) for all
    \(\vec w \in \WW\).  Why is
    \(\vec x = \sum_{i=1}^n \inner{\vec v, \vec u_i} \vec u_i\)?  This
    vector \(\vec x\) is the orthogonal projection of \(\vec v\) onto
    the subspace \(\WW\); see Section 7.3.

    \begin{solution}

    \end{solution}

  \item Deduce Bessel's inequality (5.7.5) from (5.9.1).

    \begin{solution}

    \end{solution}
  \end{enumerate}

\item[P.5.7]
  Let \(\vec u_1, \vec u_2, \dots, \vec u_n\) be an orthonormal system
  in an \(\FF\)--inner product space \(\VV\), let
  \(\UU = \spn\set{\vec u_1, \vec u_2, \dots, \vec u_n}\), and let
  \(\vec v \in \VV\).  Provide details for the following approach to
  Bessel's inequality (5.7.5):
  \begin{enumerate}
  \item If \(\vec v \in U\), why is
    \(\norm{\vec v}^2 = \sum_{i=1}^n \abs{\inner{\vec v, \vec
        u_i}}^2\)?

    \begin{solution}

    \end{solution}

  \item Suppose that \(v \not\in U\), let
    \(\WW = \spn\set{\vec u_1, \vec u_2, \dots, \vec u_n, \vec v}\),
    and apply the Gram--Schmidt process to the linearly independent
    list \(\vec u_1, \vec u_2, \dots, \vec u_n, \vec v\).  Why do you
    obtain an orthonormal basis of the norm
    \(\vec u_1, \vec u_2, \dots, \vec u_n, \vec u_{n+1}\) for \(\WW\)?

    \begin{solution}

    \end{solution}

  \item Why is
    \(\norm{\vec v}^2 = \sum_{i=1}^{n+1} \abs{\inner{\vec v, \vec
        u_i}}^2 \ge \sum_{i=1}^n \abs{\inner{\vec v, \vec u_i}}^2\)?

    \begin{solution}

    \end{solution}
  \end{enumerate}
\end{description}
\end{document}